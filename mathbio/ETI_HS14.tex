\documentclass[18pt,a4paper]{scrreprt}
\usepackage[utf8]{inputenc}
\usepackage[right=2cm,left=2cm,top=2cm]{geometry}
\usepackage{amsmath}
\usepackage{amsfonts}
\usepackage{amssymb}
\usepackage{listings}
\usepackage{color}
\usepackage{textcomp}
\definecolor{listinggray}{gray}{0.9}
\definecolor{lbcolor}{rgb}{0.9,0.9,0.9}
\usepackage{tikz}
\usetikzlibrary{automata,positioning}
\usetikzlibrary{arrows}
\usetikzlibrary{trees}
\tikzset{
  treenode/.style = {align=center, inner sep=0pt, text centered,
    },
  arn_n/.style = {treenode, circle, black,  draw=black,
    fill=white, text width=1.5em},% arbre rouge noir, noeud noir
}
\newcommand*\circled[1]{\tikz[baseline=(char.base)]{
            \node[shape=circle,draw,inner sep=1pt] (char) {#1};}}
%\author{kyriakosschwarz}
\title{Mathematik für Biologie}

\newenvironment{rcases}
  {\left.\begin{aligned}}
  {\end{aligned}\right\rbrace}

%Math
\usepackage{amsmath}
\usepackage{amsfonts}
\usepackage{amssymb}
\usepackage{amsthm}
\usepackage{ulem}

\usepackage{amsmath}
\usepackage{graphicx}
\usepackage{textcomp}

%\usepackage{relsize}

%PageStyle
%\usepackage[ngerman]{babel} % deutsche Silbentrennung
\usepackage[utf8]{inputenc} 
\usepackage{fancyhdr, graphicx}

\usepackage{parskip}
\setlength{\textwidth}{17cm}
\setlength{\oddsidemargin}{-0.5cm}


% Shortcommands
\newcommand{\Bold}[1]{\textbf{#1}} %Boldface
\newcommand{\Kursiv}[1]{\textit{#1}} %Italic
\newcommand{\T}[1]{\text{#1}} %Textmode
\newcommand{\Nicht}[1]{\T{\sout{$ #1 $}}} %Streicht Shit durch

%Arrows
\newcommand{\lra}{\leftrightarrow} 
\newcommand{\ra}{\rightarrow}
\newcommand{\la}{\leftarrow}
\newcommand{\lral}{\longleftrightarrow}
\newcommand{\ral}{\longrightarrow}
\newcommand{\lal}{\longleftarrow}
\newcommand{\Lra}{\Leftrightarrow}
\newcommand{\Ra}{\Rightarrow}
\newcommand{\La}{\Leftarrow}
\newcommand{\Lral}{\Longleftrightarrow}
\newcommand{\Ral}{\Longrightarrow}
\newcommand{\Lal}{\Longleftarrow}

\newcommand{\ts}{\textsuperscript}

%Mine(new)
\newcommand{\tab}{\hspace*{2em}}
\newcommand{\cmark}{\ding{51}}%
\newcommand{\xmark}{\ding{55}}%

\usepackage{amssymb}% http://ctan.org/pkg/amssymb
\usepackage{pifont}% http://ctan.org/pkg/pifont

%Metadata
%\fancyfoot[C]{If you use this documentation for a exam, you should offer a beer to the authors!}
\title{
	\vspace{5cm}
	Mathematik für Biologie \\
}
\author{Uni Bern}
\date{HS 2015}



\begin{document}

% Titelbild
\maketitle
\thispagestyle{fancy}
\newpage

% Inhaltsverzeichnis
%\pagenumbering{Roman}
\tableofcontents	  	
\newpage

%\setcounter{page}{1}
\pagenumbering{arabic}

% Inhalt Start

\chapter{Erste Woche}


\section{Lineares Wachstum}

\uline{Bsp}: Ein Baum wächst 20cm pro Jahr.

rekursiv (indirekte Berechnung): $H(x) = H(x-1) + 20$

explizit (direkte Berechnung): $H(x) = 20 \cdot x$

$x,n \in \mathbb{N}$, wobei $H(x)$ die Höhe des Baums nach $x$ Jahren in cm.

$H_n  = H(n) = 20 \cdot n$

\rule{\textwidth}{0.4mm}\\

\uline{Allgemeines diskretes lineares Wachstums Modell (WM)}:

rekursiv: $N_n = N_{n-1} + a$

explizit: $N_n = N_0 + a \cdot n$

$a \in \mathbb{R}, n \in \mathbb{N}$

$a > 0$: $N_n$ zunehmend 

$a < 0$: $N_n$ abnehmend

$a = 0$: $N_n$ konstant

$N: \mathbb{N} \rightarrow \mathbb{R}$ Folge (ist eine Funktion / Abbildung)

\rule{\textwidth}{0.4mm}\\

Vom rekursiven zum expliziten:

$N_{n+1} = N_n + 1a = N_{n-1} + a + a = N_{n-1} + 2a = N_{n-2} + a + 2a = N_{n-2} + 3a = ... = N_0 + (n + 1) a$

\uline{Beispiel} (Dolbearsche Gesetz)

$T_n = 1/7n + 40/9$

$T_n$: Temperatur gemessen in \textdegree C

$n$: die Anzahl der Zirplaute in einer Minute

$n = 7$: $T = ... = 5.\overline{4}$ 

$n = 14$: $T = ... = 6.\overline{4}$

...

$n = 105$: $T = ... = 19.\overline{4}$

Bereich: $5$\textdegree C - $30$\textdegree C

\uline{Beispiel}: Gewicht einer Insektenlarve zu jeder vollen Stunde:

$G(n) = 0.01n + 1, n \in \mathbb{N}$

$G(t) = 0.01t + 1, t \in \mathbb{R+}$

\rule{\textwidth}{0.4mm}\\

\uline{Allgemeines kontinuierliches lineares Wachstums Modell (WM)}:

$N_t = N_0 + t \cdot a, t \in \mathbb{R+}$

$N: \mathbb{R} \rightarrow \mathbb{R}$, $t \mapsto N(t)$ Funktion/Abbildung

\rule{\textwidth}{0.4mm}\\

Wachstumsrate (Wachstum \uline{relativ}) zur Gesamtgrösse)

diskret: $r_n = \frac{\textcolor{red}{N_{n+1} - N_n}}{N_n} = \frac{a}{N_n} = \frac{a}{N_0 + n \cdot a}$

$\textcolor{red}{N_{n+1} - N_n} = \frac{N_{n+1} - N_n}{(n+1) - n}$

kontinuierlich: $r = \frac{N'(t)}{N(t)} = \frac{(N_0 + t \cdot a)'}{N_0 + t \cdot a} = \frac{a}{N_0 + t \cdot a}, t \in \mathbb{R+}$

$\frac{N(t + \Delta t) - N(t)}{(t + \Delta t) - t}$

\rule{\textwidth}{0.4mm}\\

\section{Exponentielles Wachstum}

Beispiel (Zellteilung)

Eine Zelle teile sich zweimal pro Stunde

$N(n)$: die Anzahl Zellen nach $n$ Stunden

$N_0 = 1, N_1 = 4, N_2 = 16, N_3 = 64, ...$

rekursiv: $N_n = 4N_{n-1}, n = 1, 2, 3, 4, ...$

explizit: $N_n = 4(4N_{n-2}) = 4^2N_{n-2} = ... = 4^nN_0 = 4^n$

\rule{\textwidth}{0.4mm}\\







\end{document}