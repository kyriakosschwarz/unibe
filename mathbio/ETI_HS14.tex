\documentclass[18pt,a4paper]{scrreprt}
\usepackage[utf8]{inputenc}
\usepackage[right=2cm,left=2cm,top=2cm]{geometry}
\usepackage{amsmath}
\usepackage{amsfonts}
\usepackage{amssymb}
\usepackage{listings}
\usepackage{color}
\usepackage{textcomp}
\definecolor{listinggray}{gray}{0.9}
\definecolor{lbcolor}{rgb}{0.9,0.9,0.9}
\usepackage{tikz}
\usetikzlibrary{automata,positioning}
\usetikzlibrary{arrows}
\usetikzlibrary{trees}
\tikzset{
  treenode/.style = {align=center, inner sep=0pt, text centered,
    },
  arn_n/.style = {treenode, circle, black,  draw=black,
    fill=white, text width=1.5em},% arbre rouge noir, noeud noir
}
\newcommand*\circled[1]{\tikz[baseline=(char.base)]{
            \node[shape=circle,draw,inner sep=1pt] (char) {#1};}}
%\author{kyriakosschwarz}
\title{Mathematik für Biologie}

%\newenvironment{rcases}
 % {\left.\begin{aligned}}
  %{\end{aligned}\right\rbrace}

%Math
\usepackage{amsmath}
\usepackage{amsfonts}
\usepackage{amssymb}
\usepackage{amsthm}
\usepackage{ulem}
\usepackage[makeroom]{cancel}

\usepackage{amsmath}
\usepackage{graphicx}
\usepackage{textcomp}
\usepackage{mathtools}

%\usepackage{relsize}

%PageStyle
%\usepackage[ngerman]{babel} % deutsche Silbentrennung
\usepackage[utf8]{inputenc} 
\usepackage{fancyhdr, graphicx}

\usepackage{parskip}
\setlength{\textwidth}{17cm}
\setlength{\oddsidemargin}{-0.5cm}


% Shortcommands
\newcommand{\Bold}[1]{\textbf{#1}} %Boldface
\newcommand{\Kursiv}[1]{\textit{#1}} %Italic
\newcommand{\T}[1]{\text{#1}} %Textmode
\newcommand{\Nicht}[1]{\T{\sout{$ #1 $}}} %Streicht Shit durch

%Arrows
\newcommand{\lra}{\leftrightarrow} 
\newcommand{\ra}{\rightarrow}
\newcommand{\la}{\leftarrow}
\newcommand{\lral}{\longleftrightarrow}
\newcommand{\ral}{\longrightarrow}
\newcommand{\lal}{\longleftarrow}
\newcommand{\Lra}{\Leftrightarrow}
\newcommand{\Ra}{\Rightarrow}
\newcommand{\La}{\Leftarrow}
\newcommand{\Lral}{\Longleftrightarrow}
\newcommand{\Ral}{\Longrightarrow}
\newcommand{\Lal}{\Longleftarrow}

\newcommand{\ts}{\textsuperscript}

%Mine(new)
\newcommand{\tab}{\hspace*{2em}}
\newcommand{\cmark}{\ding{51}}%
\newcommand{\xmark}{\ding{55}}%

\usepackage{amssymb}% http://ctan.org/pkg/amssymb
\usepackage{pifont}% http://ctan.org/pkg/pifont

%Metadata
%\fancyfoot[C]{If you use this documentation for a exam, you should offer a beer to the authors!}
\title{
	\vspace{5cm}
	Mathematik für Biologie \\
}
\author{Uni Bern}
\date{HS 2015}



\begin{document}

% Titelbild
\maketitle
\thispagestyle{fancy}
\newpage

% Inhaltsverzeichnis
%\pagenumbering{Roman}
\tableofcontents	  	
\newpage

%\setcounter{page}{1}
\pagenumbering{arabic}

% Inhalt Start

\chapter{Erste Woche}


\section{Lineares Wachstum}

\uline{Bsp}: Ein Baum wächst 20cm pro Jahr.

rekursiv (indirekte Berechnung): $H(x) = H(x-1) + 20$

explizit (direkte Berechnung): $H(x) = 20 \cdot x$

$x,n \in \mathbb{N}$, wobei $H(x)$ die Höhe des Baums nach $x$ Jahren in cm.

$H_n  = H(n) = 20 \cdot n$

\rule{\textwidth}{0.4mm}\\

\uline{Allgemeines diskretes lineares Wachstums Modell (WM)}:

rekursiv: $N_n = N_{n-1} + a$

explizit: $N_n = N_0 + a \cdot n$

$a \in \mathbb{R}, n \in \mathbb{N}$

$a > 0$: $N_n$ zunehmend 

$a < 0$: $N_n$ abnehmend

$a = 0$: $N_n$ konstant

$N: \mathbb{N} \rightarrow \mathbb{R}$ Folge (ist eine Funktion / Abbildung)

\rule{\textwidth}{0.4mm}\\

Vom rekursiven zum expliziten:

$N_{n+1} = N_n + 1a = N_{n-1} + a + a = N_{n-1} + 2a = N_{n-2} + a + 2a = N_{n-2} + 3a = ... = N_0 + (n + 1) a$

\uline{Beispiel} (Dolbearsche Gesetz)

$T_n = 1/7n + 40/9$

$T_n$: Temperatur gemessen in \textdegree C

$n$: die Anzahl der Zirplaute in einer Minute

$n = 7$: $T = ... = 5.\overline{4}$ 

$n = 14$: $T = ... = 6.\overline{4}$

...

$n = 105$: $T = ... = 19.\overline{4}$

Bereich: $5$\textdegree C - $30$\textdegree C

\uline{Beispiel}: Gewicht einer Insektenlarve zu jeder vollen Stunde:

$G(n) = 0.01n + 1, n \in \mathbb{N}$

$G(t) = 0.01t + 1, t \in \mathbb{R+}$

\rule{\textwidth}{0.4mm}\\

\uline{Allgemeines kontinuierliches lineares Wachstums Modell (WM)}:

$N_t = N_0 + t \cdot a, t \in \mathbb{R+}$

$N: \mathbb{R} \rightarrow \mathbb{R}$, $t \mapsto N(t)$ Funktion/Abbildung

\rule{\textwidth}{0.4mm}\\

Wachstumsrate (Wachstum \uline{relativ} zur Gesamtgrösse)

diskret: $r_n = \frac{\textcolor{red}{N_{n+1} - N_n}}{N_n} = \frac{a}{N_n} = \frac{a}{N_0 + n \cdot a}$

$\textcolor{red}{N_{n+1} - N_n} = \frac{N_{n+1} - N_n}{(n+1) - n}$

kontinuierlich: $r = \frac{N'(t)}{N(t)} = \frac{(N_0 + t \cdot a)'}{N_0 + t \cdot a} = \frac{a}{N_0 + t \cdot a}, t \in \mathbb{R+}$

$\frac{N(t + \Delta t) - N(t)}{(t + \Delta t) - t}$

\rule{\textwidth}{0.4mm}\\

\section{Exponentielles Wachstum}

Beispiel (Zellteilung)

Eine Zelle teile sich zweimal pro Stunde

$N(n)$: die Anzahl Zellen nach $n$ Stunden

$N_0 = 1, N_1 = 4, N_2 = 16, N_3 = 64, ...$

rekursiv: $N_n = 4N_{n-1}, n = 1, 2, 3, 4, ...$

explizit: $N_n = 4(4N_{n-2}) = 4^2N_{n-2} = ... = 4^nN_0 = 4^n$

\rule{\textwidth}{0.4mm}\\

\uline{Allgemeines diskretes exponentielles Wachstums Modell (WM)}:

rekursiv: $N_n = b \cdot N_{n-1}, b \in \mathbb{R+}$

$0 < b < 1$: $N_n$ abnehmend

$b > 1$: $N_n$ zunehmend

$b = 1$: $N_n$ konstant

$b = \frac{N_n}{N_{n-1}} \textcolor{red}{\frac{\leftrightarrow y}{\leftrightarrow x}}$ \textcolor{red}{Gleichung einer Gerade durch den Ursprung mit Steigung $b$}

explizit: $N_n = b^n \cdot N_0, b \in \mathbb{R+}$

$log(N_n) = log(b^n \cdot N_0)$

$log(N_n) = log(b^n) + log(N_0)$

$log(N_n) = n\cdot log(b) + log(N_0)$

In der $log$ Skala erscheint exponentielles Wachstum linear

\uline{Zellteilung}:

$log(N_n) = n\cdot log(4) + log(1)$

$log(N_n) = n\cdot log(4)$

\rule{\textwidth}{0.4mm}\\

\chapter{Zweite Woche}

\section{Wachstumsrate}

\uline{Diskrete Wachstumsrate}

$r_n = \frac{N_{n+1} - N_n}{N_n} = \frac{\frac{N_{n+1} - N_n}{n+1 -n}}{N_n}$

Im diskreten exponentiellen Modell:

$r_n = \frac{b\cdot N_n - N_n}{N_n} = b-1$ konstant

(diskretes exponentielles Wachstum $\Rightarrow$ $r_n$ konstant)

\uline{Frage}: $r_n$ konstant $\xRightarrow{?}$ $N_n$ wächst exponentiell

$r_n$ konst. $\Rightarrow r_n = c \Rightarrow \frac{N_{n+1} - N_n}{N_n} = c \Rightarrow N_{n+1} - N_n = c\cdot N_n \Rightarrow N_{n+1} = (c+1) N_n$ (diskr. exp. W.)

\uline{Fazit}: Ein diskretes exponentielles WM ist durch eine konstante Wachstumsrate charakterisiert.

\rule{\textwidth}{0.4mm}\\

\uline{Bemerkungen}

\begin{itemize}
	\item $b-1 > 0$ exp. Wachstum
	\item $b-1 < 0$ exp. Zerfall
	\item $b-1 = 0$ $N_n$ bleibt konstant
\end{itemize}

\rule{\textwidth}{0.4mm}\\

\section{Kontinuierliche Wachstumsrate}

$N: \mathbb{R+} \rightarrow \mathbb{R}$ reele Funktion $N(t)$

durchschnittliche Änderung pro Zeiteinheit im Bereich von $t$ bis $t+\Delta t$, wobei $\Delta t >0: \frac{N(t+\Delta t) - N(t)}{\Delta t}$

\uline{momentane Wachstumsgeschwindigkeit zum Zeitpunkt $t$}

$lim_{\Delta t \rightarrow 0}\frac{N(t+\Delta t) - N(t)}{\Delta t} = N'(t)$

konkav? konvex?

\rule{\textwidth}{0.4mm}\\

\uline{Kontinuierliche Wachstumsrate}

$r(t) = \frac{N'(t)}{N(t)}$

\uline{kont. exp. WM}

Konstante Wachstumsrate

$\frac{N'(t)}{N(t)} = c$

Differentialgleichung (DG): gesucht ist die Funktion $N(t)$. In dieser Gleichung finden wir auch die Ableitung $N'(t)$ von $N(t)$

$N'(t) = c\cdot N(t)$

$N(t) = e^{c\cdot t}$ ist eine Lösung der DG. 

Überprüfung: $N'(t) = (e^{c t})' = c\cdot e^{c t} = c \cdot N(t)$ \checkmark

Anfangsbedingung: $N(0) = N_0$

$N(t) = e^{c t} \cdot N_0$

$t = 0$: $1\cdot N_0 = N_0$ \checkmark

$N'(t) = (N_0 \cdot e^{c t})' = c\cdot(N_0 \cdot e^{c t}) = c \cdot N(t)$ \checkmark

ist eine (der vielen??) Lösung der DG mit Anfangsbedingung $N(0) = N_0$

\rule{\textwidth}{0.4mm}\\

\uline{Kontinuierliches exponentielles Wachstumsmodell}

$\frac{N'(t)}{N(t)} = r$ \tab DG

$N(t) = e^{rt}$ ist eine Lösung der DG

$N(0) = N_0$ Anfangsbedingung

$N(t) = N_0 \cdot e^{rt}$ ist eine Lösung der DG mit Anfangsbedingung $N(0) = N_0$

Konvex: Steigung nimmt zu oder $\underbrace{N''(t)}_{= r^2 \cdot N_0 \cdot e^{rt}} > 0$

Eindeutig? \uline{Antwort}: 

Sei $X(t)$ eine beliebige Lösung der DG

D.h. $\frac{X'(t)}{X(t)} = r \Leftrightarrow X'(t) = r \cdot X(t)$

\tab \tab \tab \tab $N'(t) = r \cdot N(t)$ 

$(\frac{X(t)}{e^{rt}})' = (e^{-rt} \cdot X(t))' = (e^{-rt})' \cdot X(t) + e^{-rt} \cdot (X(t))' = -r\cdot e^{-rt} \cdot X(t) + e^{-rt} \cdot X'(t) = -r\cdot e^{-rt} \cdot X(t) + e^{-rt} \cdot r\cdot X(t) = 0$

d.h. $\frac{X(t)}{e^{rt}} = c \Leftrightarrow X(t) = c\cdot e^{rt}$ die exponentielle Funktion ist eindeutig

Anfangsbedingung $X(0) = N_0$

$c \cdot e^{r\cdot 0} = N_0$

$c = N_0$

d.h. $X(t) = N_0 \cdot e^{rt}$ ist eine eindeutige Lösung der DG $X'(t) = r\cdot X(t)$ mit A.B. $X(0) = N_0$

\rule{\textwidth}{0.4mm}\\

\section{Logarithmisches Wachstum}

$\frac{55-50}{50} = \frac{5}{50} = 0.1 = 10\%$

$\frac{5005-5000}{5000} = \frac{5}{5000} = 0.001 = 0.1\%$

Ein relativer Gewichtsunterschied von $\sim 2\%$ eines in einer ruhenden Hand gehaltenen Gegenstands wird erkannt. D.h. bei $50$gr ein Gewichtsunterschied von $1$gr.

\rule{\textwidth}{0.4mm}\\

\uline{Gesetz von Weber und Fechner}

(Beziehung zwischen Stimulus und Wahrnehmung)

Unsere Wahrnehmung einer Intensitätsänderung ist proportional zur relativen Änderung des Stimulus.

Mathematisch formuliert:

$S$ die Intensität des Stimulus

$\Delta S$ die Änderung dieser Intensität

$W(S)$ die (von $S$ abh.) Stärke der Wahrnehmung

$\Delta W(S) = K \cdot \frac{\Delta S}{S}$, $K$ Konstante

umgeformt: $\frac{\Delta W(S)}{\Delta S} = \frac{K}{S}$

für klein werd. $\Delta S$: $\lim_{\Delta S\to 0} \frac{\Delta W(S)}{\Delta S} = W'(S) = \frac{K}{S}$

gesucht ist $W(S)$. In der Gleichung taucht $W'(S)$ auf $\rightarrow$ D.G.

$\int K \cdot \frac{1}{S} dS = K \cdot \int \frac{1}{S} dS = K \cdot ln(S) + c$

$(K\cdot ln(S))' = K \cdot \frac{1}{S} = \frac{K}{S}$

\uline{Definition der Ableitung}: 

$W'(S) = \lim_{\Delta S\to 0} \frac{W(S + \Delta S) - W(S)}{\Delta S} = \lim_{\Delta S\to 0} \frac{\Delta W(S)}{\Delta S}$

\rule{\textwidth}{0.4mm}\\

$W(S) = K \cdot ln(S) + c$

$S_0$: die grösste Intensitätsgrenze, bei der keine Wahrnehmung möglich ist.

$W(S_0) = K \cdot ln(S_0) + c = 0$

$c = -K \cdot ln(S_0)$

$W(S) = K \cdot ln(S) - K \cdot ln(S_0)$

$W(S) = K \cdot ln(\frac{S}{S_0})$

$W(S) = a \cdot ln(b\cdot S)$, $a,b$ konst.

\uline{Beispiel}:

$a =1, b=2$

$W(S) = ln(2S)$

$S = 1/2$

$W(S) = ln(1) = 0$

\uline{Darstellung}:

$W(S) = a\cdot ln(bS) = a\cdot (ln(b) + ln(S)) = a\cdot ln(b) + a\cdot ln(S)$

$log_{10}(S) = \frac{ln(S)}{ln(10)}$

$\underbrace{a\cdot ln(b)}_{\substack{\text{konst.}\\y\text{-Achsenabschnitt}}} + \underbrace{a\cdot ln(10)}_{\text{Steigung}} \cdot log_{10}(S)$

\rule{\textwidth}{0.4mm}\\

\chapter{Dritte Woche}

\section{Logarithmisches Wachstum (vort.)}

$W(S) = a \cdot ln(b\cdot S)$, $a,b$ konst.

$a = 1, b =2$: $W(S)= ln(2S)$

$(ln = log_e, e \simeq 2.7$ die Eulersche Zahl\\ $ln(x) = y \Leftrightarrow e^y = x)$

$W(1) = ln(2) \simeq 0.69$

$W(2000) = ln(4000) \simeq 8.3$

$W(10000) = ln(20000) \simeq 9.9$

$W(S) = a\cdot ln(b) + a\cdot ln(S) = \underbrace{a\cdot ln(b)}_{\substack{\text{konst.}\\y\text{-Achsenabschnitt}}} + \underbrace{a\cdot ln(10)}_{\text{Steigung}} \cdot log_{10}(S)$

$W(S) = ln(2) + ln(10) \cdot log_{10}(S) \simeq 0.69 + 2.3 \cdot log_{10}(S)$

$ln(S) \stackrel{?}{=} ln(10) \cdot log_{10}(S)$

Hinweis: $10^{log_{10}(S)} = S$

$ln(10^{log_{10}(S)}) = ln(S)$

$log_{10}(S) \cdot ln(10) = ln(S)$

\rule{\textwidth}{0.4mm}\\

Eindeutigkeit der Lösung der DG $W'(S) = \frac{K}{S}$ mit A.B. $W(S_0)= 0$

Wir haben schon die Lösung $W(S)= k\cdot ln(\frac{S}{S_0})$ gefunden

Sei $X(S)$ eine beliebige Lösung, d.h. $X'(S)= \frac{K}{S}$ und $X(S_0) = 0$

$W'(S) - X'(S) = 0 \Leftrightarrow W(S) - X(S) = c$ für eine Konstante $c$ $(1)$

für $S = S_0$: $W(S_0) - X(S_0) = c \Leftrightarrow 0-0 = c \Leftrightarrow c =0$ $(2)$

$\stackrel{(1)(2)}{\Rightarrow} X(S) = W(S)$  
 
\rule{\textwidth}{0.4mm}\\

\section{Logistisches Wachstum}

exp. Wachstum $\frac{N'(t)}{N(t)} = \underbrace{r}_{\text{konst.}} + \underbrace{a}_{\text{neg. konst.}}\cdot N(t)$ $(1)$

Kommt das Wachstum zum Stehen?

$0 = r + a \cdot \underbrace{K}_{\text{obere Schranke}}$

d.h.: $a =\frac{-r}{K}$ $(2)$ 

$\stackrel{(1)(2)}{\Rightarrow} \frac{N'(t)}{N(t)} = r - \frac{r}{K} \cdot N(t) = r(1 - \frac{N(t)}{K})$

$\frac{N'(t)}{N(t)} = r(1 - \frac{N(t)}{K})$ DG, $N(t) = ?$

\rule{\textwidth}{0.4mm}\\

exp. WM $\frac{N'(t)}{N(t)} = r$, eine obere Schranke $K$ einführen

$\frac{N'(t)}{N(t)} = r + \underbrace{a}_{\text{neg. konst.}} \cdot N(t)$

$\frac{N'(t)}{N(t)} = r - \frac{r}{K} \cdot N(t) = r (1 - \frac{N(t)}{K})$ DG $(1)$

A.B. $N(0) = N_0$

\uline{Qualitative Analyse von} $N'(t) = r\cdot N(t) \cdot (1 - \frac{N(t)}{K})$ mit $N_0 < K$

\uline{Anfangsphase}: $N(t)$ ist relativ klein im Vergleich zu $K$ dann ist $\frac{N^2(t)}{K}$ relativ klein

d.h. $N'(t) = r \cdot N(t) - r \cdot \frac{N^2(t)}{K} \simeq r\cdot N(t)$

d.h. $N(t)$ wächst ungefähr exponentiell

z.B. $N_0 = 10$, $K = 10000$

\uline{Mittlere Wachstumsphase} der Term $\frac{N^2(t)}{K}$ ist wichtiger. Dach Wachstum wird abgebremst aber die Population wächst immer noch.

$N'(t) = \underbrace{r\cdot N(t)}_{>0} \underbrace{1-\frac{N(t)}{K}}_{>0}$

\uline{Abflachungsphase} $N(t)$ nähert sich der Zahl $K$ an und das Wachstum kommt fast zum stehen. 

$\underbrace{N'(t)}_{\simeq 0} = r\cdot N(t) 1-\underbrace{\frac{N(t)}{K}}_{\simeq 1}$ 

\uline{Behauptung}: $K$ ist die kleinste obere Schranke für $N(t)$

\uline{Begründung}

1. $K$ ist eine obere Schranke

Es gibt kein $\bar{t}$ so dass $N(t)$ zu diesem Zeitpunkt $\bar{t}$ über $K$ hinauswächst

Gäbe es einen solchen Zeitpunkt $\bar{t}$, dann würde Folgendes gelten:

$N(\bar{t}) = K$

$N'(\bar{t}) > 0$

$N'(\bar{t}) = r\cdot K (1-\frac{K}{K}) = 0$ Wiederspruch

Unsere Annahme ist falsch. Es gibt keinen solchen Zeitpunkt.

d.h. $K$ ist eine obere Schranke für $N(t)$

2. $K$ ist die \uline{kleinste} obere Schranke

Solange $0<N(t)<K$ gilt $N'(t)> 0$

\uline{Gleichgewichtszustände}

$N'(t) = 0 \Leftrightarrow r\cdot N(t) (1-\frac{N(t)}{K}) =0$

a. $N(t) = 0$

b. $N(t) = K$

d.h. falls $N_0 =0$ oder $N_0 = K$ dann bleibt $N(t)$ konstant.

\uline{Explizite Lösung von (1)}

Gesucht ist $N(t)$

\uline{Ansatz}: $N(t) = \frac{e^{rt}}{f)t}$

$N'(t) = \frac{r\cdot e^{rt}\cdot f(t) - e^{rt} \cdot f'(t)}{(f(t))^2}$

$\stackrel{(1)}{=} r\cdot \frac{e^{rt}}{f(t)}(1-\frac{e^{rt}}{f(t)\cdot K})=r\cdot N(t)(1-\frac{N(t)}{K})$

$\Leftrightarrow r\cdot e^{rt} \cdot f(t) - e^{rt} \cdot f'(t) = r\cdot e^{rt} \cdot f(t) (1-\frac{e^{rt}}{k\cdot f(t)})$

$\Leftrightarrow \cancel{r\cdot e^{rt} \cdot f(t)} - e^{rt} \cdot f'(t) = \cancel{r\cdot e^{rt} \cdot f(t)} - r\cdot \frac{(e^{rt})^2}{K}$

$\Leftrightarrow f'(t) = \frac{r\cdot e^{rt}}{K}$

$f(t) = \frac{1}{K} e^{rt} + c$

(überpr. $f'(t) = \frac{1}{K} \cdot r \cdot e^{rt} = \frac{r\cdot e^{rt}}{K} \checkmark$)

$N(t)= \frac{e^{rt}}{\frac{1}{K}\cdot e^{rt} + c}$ (2)

$\Leftrightarrow \frac{1}{N_0} = \frac{1}{K} + c$

$N(0) = \frac{1}{\frac{1}{K} + c} = N_0$

$\Leftrightarrow c = \frac{1}{N_0} - \frac{1}{K} = \frac{K-N_0}{K\cdot N_0}$ (3)













\end{document}